\section{Problem 3}
\textit{Süli-Mayers: Ex. 1.8, 2.8, 4.8}

\subsection{Exercise 1.8}
\textit{Suppose that the function $f$ has a continuous second derivative, that $f(\xi) = 0$, and that in the interval $[X, \xi]$, with $X < \xi$, $f'(x) > 0$ and $f''(x) < 0$. Show that the Newton iteration, starting from any $x_0$ in $[X, \xi]$, converges to $\xi$.}

As $f'(x) > 0$ for all $x \in [X, \xi]$, $f$ must be absolutely monotonically increasing over the interval. As $f(\xi) = 0$, this means that $f(x_0) < 0$ for all $x_0 \in [X, \xi)$.

The Newton iteration states that $x_{n+1} = x_n - \frac{f(x_n)}{f'(x_n)}$. For $x_0 = \xi$, observe that $f(\xi) = 0$, thus $f(x_0) = 0$, thus
\begin{equation*}
    x_{n+1} = x_n - \frac{f(x_n)}{f'(x_n)} = x_n 
\end{equation*}
\begin{equation*}
    \lim_{n\to\infty} x_n = x_0 = \xi
\end{equation*}

As $f(x_n) < 0$ and $f'(x_n) > 0$ for all $x_n \in [X, \xi]$, we have that $\frac{f(x_n)}{f'(x_n)} < 0$ for all $x_n \in [X, \xi)$. This means that, for $x_n \in [X, \xi)$:
\begin{align*}
    x_{n+1} &= x_n - \frac{f(x_n)}{f'(x_n)} \\
        &= x_n + \abs{\frac{f(x_n)}{f'(x_n)}} > x_n
\end{align*}
This means that for $x_0 \in [X, \xi)$:
\begin{equation*}
    \lim_{n\to\infty} x_n = \xi
\end{equation*}

As $\lim_{n\to\infty} x_n = \xi$ for both $x_0 = \xi$ and $x_0 \in [X, \xi)$, it follows that
\begin{equation*}
    \lim_{n\to\infty} x_n = \xi
\end{equation*}
for all $x_0 \in [X, \xi]$, and that the Newton iteration converges to $\xi$ for any $x_0 \in [X, \xi]$.


\pagebreak
\subsection{Exercise 2.8}
\textit{(i) Show that, for any vector $\mathbf{v} = (v_1, ..., v_n)^T \in \mathbb{R}^n$, $\norm{\mathbf{v}}_\infty \le \norm{\mathbf{v}}_2$ and $\norm{\mathbf{v}}_2^2 \le \norm{\mathbf{v}}_1 \norm{\mathbf{v}}_\infty$.}

\textit{In each case give an example of a nonzero vector $\mathbf{v}$ for which equality is attained. Deduce that $\norm{\mathbf{v}}_\infty \le \norm{\mathbf{v}}_2 \le \norm{\mathbf{v}}_1$. Show also that $\norm{\mathbf{v}}_2 \le \sqrt{n}\norm{\mathbf{v}}_\infty$.}

Observe that 
\begin{equation*}
    \norm{v}_\infty = \max_i \abs{v_i} \ge v_i \; \forall i,
\end{equation*}
thus
\begin{equation*}
    \norm{v}_2 = \sqrt{\sum_{i=1}^n \abs{v_i}^2} \le \sqrt{\sum_{i=1}^n \norm{v}_\infty^2} = \sqrt{n\norm{v}_\infty^2} = \sqrt{n}\norm{v}_\infty
\end{equation*}
and
\begin{equation*}
    \norm{v}_\infty = \sqrt{(\max_i \abs{v_i})^2} \le \sqrt{\sum_{i=1}^n \abs{v_i}^2} = \norm{v}_2
\end{equation*}

If $n = 1$, then $v_i = \norm{v}_\infty \; \forall i$, giving $\norm{v}_2 = \sqrt{n}\norm{v}_\infty$. It follows that $\norm{v}_2 = \sqrt{1}\norm{v}_\infty = \norm{v}_\infty$, thus any vector with length 1 (such as $v = (\pi)$) is an example where equality is attained. This is proven as $\norm{(\pi)}_\infty = max([\pi]) = \pi$, and $\norm{(\pi)}_2 = \sqrt{\pi^2} = \pi$.

If $n > 1$, let $j$ be the index there $v_j = \max_i \abs{v_i}$. It then follows that $(\max_i \abs{v_i})^2 = \abs{v_j}^2 < \sum_{i=1}^n \abs{v_i}^2$ (as $\sum_{i=1}^n \abs{v_i}^2 = \abs{v_j}^2 + \sum_{i \ne j} \abs{v_i}^2$), thus inequality is attained and $\norm{v}_\infty \le \norm{v}_2$ for all vectors $v \in \mathbb{R}^n$.

Observe that
\begin{align*}
    \norm{v}_2^2 = \sqrt{\sum_{i=1}^n \abs{v_i}^2}^2 = \sum_{i=1}^n \abs{v_i}^2 &\le \sum_{i=1}^n \abs{v_i} \max_i \abs{v_i} \\
    &= \sum_{i=1}^n \abs{v_i} \norm{v}_\infty = n \norm{v}_\infty \sum_{i=1}^n \abs{v_i} = n \norm{v}_\infty \norm{v}_1
\end{align*}

If $\norm{v}_\infty = v_i \; \forall i$, then it follows that $\norm{v}_2^2 = \norm{v}_1 \norm{v}_\infty$, thus any vector where this is true (such as any vector with length 1, such as $v = (\pi)$) is an example where equality is attained. This is proven as $\norm{(\pi)}_2^2 = \sqrt{\pi^2}^2 = \pi^2$, and $\norm{(\pi)}_1 * \norm{(\pi)}_\infty = \pi * \pi = \pi^2$.

If $\norm{v}_\infty > v_i$ for any $i$, then it follows that $\norm{v}_2^2 < \norm{v}_1 \norm{v}_\infty$. This can be proven by the vector $v = (1, 2)$, where $\norm{v}_2^2 = \sqrt{1 + 4}^2 = 5$, and $\norm{v}_1 * \norm{v}_\infty = (1 + 2) * 2 = 6$.

It follows from this that $\norm{v}_2^2 \le \norm{v}_1 \norm{v}_\infty$ for all vectors $v \in \mathbb{R}^n$.

As $\norm{v}_\infty > 0$ and $\norm{v}_2 \ge \norm{v}_\infty$, we have that
\begin{equation*}
    \norm{v}_2 = \frac{\norm{v}_2^2}{\norm{v}_2} \le \frac{\norm{v}_1 \norm{v}_\infty}{\norm{v}_2} \le \frac{\norm{v}_1 \norm{v}_\infty}{\norm{v}_\infty} = \norm{v}_1
\end{equation*}
Combining this with the earlier answer, it follows that 
\begin{equation*}
    \norm{v}_\infty \le \norm{v}_2 \le \norm{v}_1
\end{equation*}


\pagebreak
\textit{(ii) Show that, for any matrix $A \in \mathbb{R}^{m*n}$, $\norm{A}_\infty \le \sqrt{n}\norm{A}_2$ and $\norm{A}_2 \le \sqrt{m} \norm{A}_\infty$.}

\textit{In each case give an example of a matrix $A$ for which equality is attained.}

Observe that 
\begin{equation*}
    \norm{A}_\infty = \max_{i=1}^n \sum_{j=1}^m \abs{a_{ij}}
\end{equation*}
i.e. $\norm{A}_\infty$ is the maximum row-sum in the matrix. Let $A_k$ be the row which satisfies
\begin{equation*}
    \norm{A}_\infty = \max_{i=1}^n \sum_{j=1}^m \abs{a_{ij}} = \norm{A_k}_1
\end{equation*}
i.e. the row with the maximum row-sum in $A$. Then we have
\begin{align*}
    \norm{A}_\infty = \norm{A_k}_1 &\le \sqrt{n} \norm{A_k}_2 \\
    &\le \sqrt{n} \sum_{i=1}^n \norm{A_i}_2 = \sqrt{n} \norm{A}_2
\end{align*}
where the first inequality was proven in \textit{(i)}, and the second follows from the fact that $\sum_{i=1}^n \norm{A_i}_2 = \norm{A_k}_2 + \sum_{i \ne k} \norm{A_i}_2$.

Note that equality is attained when $A$ is a square matrix with $n = 1$, such as $A = \begin{bmatrix}2\end{bmatrix}$. In this case, $\norm{A}_\infty = \max (2) = 2$, and $\sqrt{n}\norm{A}_2 = 1 * 2 = 2$.

---

Definition 2.10 gives that
\begin{equation*}
    \norm{A} = \max_{x \in \mathbb{R}^n \setminus \{0\}} \frac{\norm{Ax}}{\norm{x}}
\end{equation*}

Combining this with the proofs found in \textit{(i)}, observe that
\begin{align*}
    \norm{A}_2 = \max_x \frac{\norm{Ax}_2}{\norm{x}_2} &\le \max_x \frac{\sqrt{m}\norm{Ax}_\infty}{\norm{x}_2} \\
    &\le \max_x \frac{\sqrt{m}\norm{Ax}_\infty}{\norm{x}_\infty} = \sqrt{m} \norm{A}_\infty
\end{align*}
where the first inequality follows from $\norm{v}_2 \le \sqrt{n} \norm{v}_\infty$, as found in \textit{(i)}, and the second inequality follows from $\norm{v}_2 \ge \norm{v}_\infty$, also found in \textit{(i)}.

Equality is attained, also here, when $A$ is a square matrix with $n = 1$, such as $A = \begin{bmatrix}2\end{bmatrix}$, where $\norm{A}_2 = 2$ and $\sqrt{m} \norm{A}_\infty = 2$.


\pagebreak
\subsection{Exercise 4.8}
\textit{Suppose that $\xi = \lim_{k \to \infty} \mathbf{x}^{(k)}$ in $\mathbb{R}$. Following Definition 1.4, explain what is meant by saying that "the sequence $\mathbf{x}^{(k)}$ converges to $\xi$ linearly, with asymptotic rate $-\log_{10} \mu$", where $0 < \mu < 1$.}

When a sequence $\mathbf{x}$ converges linearly to a constant $C$ with an asymptotic rate of $\gamma$, this means that the sequence satisfies the equation
\begin{equation}
    \lim_{k \to \infty} \frac{\abs{\mathbf{x}^{(k + 1)} - C}}{\abs{\mathbf{x}^{(k)} - C}} = \gamma
\end{equation}
In this case, the sequence $\mathbf{x}$ satisfies the following equation:
\begin{equation*}
    \lim_{k \to \infty} \frac{\abs{\mathbf{x}^{(k + 1)} - \xi}}{\abs{\mathbf{x}^{(k)} - \xi}} = \log_{10} \mu
\end{equation*}


\textit{Given the vector function $\mathbf{x} \mapsto \mathbf{f(x)}$ of two real variables $x_1$ and $x_2$ defined by $f_1(x_1, x_2) = x_1^2 + x_2^2 - 2$ and $f_2(x_1, x_2) = x_1 + x_2 - 2$, show that $\mathbf{f(\xi)} = \mathbf{0}$ when $\mathbf{\xi} = (1, 1)^T$.}

\begin{align*}
    \mathbf{f}(\xi) &= \mathbf{f}(1, 1)^T \\
        &= (f_1(1, 1), f_2(1,1)) \\
        &= (1 + 1 - 2, 1 + 1 - 2) = (0, 0) = \mathbf{0}
\end{align*}

\textit{Suppose that $x_1^{(0)} \ne x_2^{(0)}$; show that one iteration of Newton's method for the solution $\mathbf{f(x) = 0}$ with starting value $\mathbf{x}^{(0)} = (x_1^{(0)}, x_2^{(0)})^T$ then gives $\mathbf{x}^{(1)} = (x_1^{(1)}, x_2^{(1)})^T$ such that $x_1^{(1)} + x_2^{(1)} = 2$.}

Newton's method states that $x^{(n + 1)} = x^{(n)} - J_f(\mathbf{x}^{(n)})^{-1} f(\mathbf{x}^{(n)})$, where $J_f(\mathbf{x}^{(n)})^{-1}$ is the inverse of the Jacobian matrix of f. We compute this as 
\begin{align*}
    J_f(\mathbf{x}^{(n)}) &=
    \begin{bmatrix}
        \frac{\partial f_1}{\partial x_1} & \frac{\partial f_1}{\partial x_2} \\
        \frac{\partial f_2}{\partial x_1} & 
        \frac{\partial f_2}{\partial x_2} \\
    \end{bmatrix} \\
    &= \begin{bmatrix}
        2x_1 & 2x_2 \\
        1 & 1 \\
    \end{bmatrix} \\
    &= \begin{bmatrix}
        \frac{1}{2x_1 - 2x_2} & \frac{-x_2}{x_1 - x_2} \\
        \frac{-1}{2x_1 - 2x_2} & \frac{x_1}{x_1 - x_2} \\
    \end{bmatrix} \\
\end{align*}

It follows that
\begin{align*}
    \mathbf{x}^{(1)} &= \mathbf{x}^{(0)} - J_f(\mathbf{x}^{(0)})^{-1} f(\mathbf{x}^{(0)}) \\
    &= (x_1^{(0)}, x_2^{(0)}) - \begin{bmatrix}
        \frac{1}{2x_1 - 2x_2} & \frac{-x_2}{x_1 - x_2} \\
        \frac{-1}{2x_1 - 2x_2} & \frac{x_1}{x_1 - x_2} \\
    \end{bmatrix} * (x_1^2 + x_2^2 - 2, x_1 + x_2 - 2)^T \\
    &= (x_1^{(0)}, x_2^{(0)}) - \begin{bmatrix}
        \frac{x_1^2 + x_2^2 - 2}{2x_1 - 2x_2} + \frac{-(x_1x_2 + x_2^2 - 2x_1)}{x_1 - x_2} \\
        \frac{-(x_1^2 + x_2^2 - 2)}{2x_1 - 2x_2} + \frac{x_1^2 + x_1x_2 - 2x_1}{x_1 - x_2} \\
    \end{bmatrix} \\
    &= (x_1^{(0)}, x_2^{(0)}) - \begin{bmatrix}
        \frac{x_1^2 + x_2^2 - 2 - 2x_1x_2 - 2x_2^2 + 4x_1}{2x_1 - 2x_2} \\
        \frac{-x_1^2 - x_2^2 + 2 + 2x_1^2 + 2x_1x_2 - 4x_1}{2x_1 - 2x_2} \\
    \end{bmatrix} \\
    &= (x_1^{(0)}, x_2^{(0)}) - \begin{bmatrix}
        \frac{x_1^2 - x_2^2 - 2x_1x_2 + 4x_1  - 2}{2x_1 - 2x_2} \\
        \frac{x_1^2 - x_2^2 + 2x_1x_2 - 4x_1  + 2}{2x_1 - 2x_2} \\
    \end{bmatrix} \\
    &= \begin{bmatrix}
        x_1 - \frac{x_1^2 - x_2^2 - 2x_1x_2 + 4x_1  - 2}{2x_1 - 2x_2} \\
        x_2 - \frac{x_1^2 - x_2^2 + 2x_1x_2 - 4x_1  + 2}{2x_1 - 2x_2} \\
    \end{bmatrix} \\
    &= \begin{bmatrix}
        \frac{2x_1^2 - 2x_1x_2 - (x_1^2 - x_2^2 - 2x_1x_2 + 4x_1 - 2)}{2x_1 - 2x_2} \\
        \frac{2x_1x_2 - 2x_2^2 - (x_1^2 - x_2^2 + 2x_1x_2 - 4x_1 + 2)}{2x_1 - 2x_2} \\
    \end{bmatrix} \\
    &= \begin{bmatrix}
        \frac{x_1^2 + x_2^2 - 4x_1  + 2}{2x_1 - 2x_2} \\
        \frac{-x_1^2 - x_2^2 + 4x_1  - 2}{2x_1 - 2x_2} \\
    \end{bmatrix} \\
\end{align*}

It is clear to see that
\begin{align*}
    x_1^{(1)} + x_2^{(1)} &= \frac{x_1^2 + x_2^2 - 4x_1  + 2}{2x_1 - 2x_2} + \frac{-x_1^2 - x_2^2 + 4x_1  - 2}{2x_1 - 2x_2} \\
    &= \frac{x_1^2 + x_2^2 - 4x_1  + 2 - x_1^2 - x_2^2 + 4x_1  - 2}{2x_1 - 2x_2} \\
    &= \frac{0}{2x_1 - 2x_2} = 0
\end{align*}

\textit{This is a different result than what I was supposed to show, which is interesting. I have not succeeded in pinpointing exactly where I was wrong...}


\textit{Determine $\mathbf{x}^{(1)}$ when $x_1^{(0)} = 1 + \alpha$, $x_2^{(0)} = 1 - \alpha$, where $\alpha \ne 0$. Assuming that $x_1^{(0)} \ne x_2^{(0)}$, deduce that Newton's method converges linearly to $(1, 1)^T$, with asymptotic rate of convergence $\log_{10}2$. Why is the convergence rate not quadratic?}

\begin{align*}
    \mathbf{x}^{(1)} &= \mathbf{x}^{(0)} - J_f(\mathbf{x}^{(0)})^{-1} f(\mathbf{x}^{(0)}) \\
    &= (1 + \alpha, 1 - \alpha) - \begin{bmatrix}
        \frac{1}{2 + 2\alpha - 2 + 2\alpha} & \frac{-1 + \alpha}{1 + \alpha - 1 + \alpha} \\
        \frac{-1}{2 + 2\alpha - 2 + 2\alpha} & \frac{1 + \alpha}{1 + \alpha - 1 + \alpha} \\
    \end{bmatrix} * (1 + 2\alpha + \alpha^2 + 1 - 2\alpha + \alpha^2 - 2, 1 + \alpha + 1 - \alpha - 2)^T \\
    &= (1 + \alpha, 1 - \alpha) - \begin{bmatrix}
        \frac{1}{4\alpha} & \frac{-1 + \alpha}{2\alpha} \\
        \frac{-1}{4\alpha} & \frac{1 + \alpha}{2\alpha} \\
    \end{bmatrix} * (2\alpha^2, 2\alpha)^T \\
    &= (1 + \alpha, 1 - \alpha) - \begin{bmatrix}
        \frac{\alpha}{2} -1 + \alpha \\
        \frac{-\alpha}{2} + 1 + \alpha \\
    \end{bmatrix} \\
    &= (1 + \alpha - \frac{\alpha}{2} + 1 - \alpha, 1 - \alpha + \frac{\alpha}{2} - 1 - \alpha) \\
    &= (2 - \frac{\alpha}{2}, - 2\alpha + \frac{\alpha}{2}) \\
\end{align*}

For the convergence rate to be quadratic, it would have to satisfy the following inequation:
\begin{equation}
    \lim_{k \to \infty} \frac{\abs{x^{(k + 1)} - (1, 1)^T}}{\abs{x^{(k)} - (1, 1)^T}^2} < C
\end{equation}
It does not, however, but it is easy to see that
\begin{equation}
    \lim_{k \to \infty} \frac{\abs{x^{(k + 1)} - (1, 1)^T}}{\abs{x^{(k)} - (1, 1)^T}} < \log_{10} 2,
\end{equation}
which gives a linear convergence to $(1, 1)^T$ with asymptotic rate of convergence $\log_{10} 2$.
